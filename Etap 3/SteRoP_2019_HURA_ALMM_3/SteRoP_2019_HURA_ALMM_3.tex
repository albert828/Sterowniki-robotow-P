% !TeX encoding = UTF-8
% !TeX spellcheck = pl_PL

% $Id:$

%Author: Wojciech Domski
%Szablon do ząłożeń projektowych, raportu i dokumentacji z steorwników robotów
%Wersja v.1.0.0
%


%% Konfiguracja:
\newcommand{\kurs}{Sterowniki robot\'{o}w}
\newcommand{\formakursu}{Projekt}

%odkomentuj właściwy typ projektu, a pozostałe zostaw zakomentowane
%\newcommand{\doctype}{Za\l{}o\.{z}enia projektowe} %etap I
%\newcommand{\doctype}{Raport} %etap II
\newcommand{\doctype}{Dokumentacja} %etap III

%wpisz nazwę projektu
\newcommand{\projectname}{Humanistycznie upo\'sledzony robot akrobatyczny}

%wpisz akronim projektu
\newcommand{\acronim}{HURA}

%wpisz Imię i nazwisko oraz numer albumu
\newcommand{\osobaA}{Albert \textsc{Lis}, 235534}
%w przypadku projektu jednoosobowego usuń zawartość nowej komendy
\newcommand{\osobaB}{Micha\l{} \textsc{Moru\'n}, 235986}

%wpisz termin w formie, jak poniżej dzień, parzystość, godzina
\newcommand{\termin}{sr TP15 }

%wpisz imię i nazwisko prowadzącego
\newcommand{\prowadzacy}{mgr in\.{z}. Wojciech \textsc{Domski}}

\documentclass[10pt, a4paper]{article}

\include{preambula}
	
\begin{document}

\def\tablename{Tabela}	%zmienienie nazwy tabel z Tablica na Tabela

\begin{titlepage}
	\begin{center}
		\textsc{\LARGE \formakursu}\\[1cm]		
		\textsc{\Large \kurs}\\[0.5cm]		
		\rule{\textwidth}{0.08cm}\\[0.4cm]
		{\huge \bfseries \doctype}\\[1cm]
		{\huge \bfseries \projectname}\\[0.5cm]
		{\huge \bfseries \acronim}\\[0.4cm]
		\rule{\textwidth}{0.08cm}\\[1cm]
		
		\begin{flushright} \large
		\emph{Skład grupy:}\\
		\osobaA\\
		\osobaB\\[0.4cm]
		
		\emph{Termin: }\termin\\[0.4cm]

		\emph{Prowadzący:} \\
		\prowadzacy \\
		
		\end{flushright}
		
		\vfill
		
		{\large \today}
	\end{center}	
\end{titlepage}

\newpage
\tableofcontents
\newpage

%Obecne we wszystkich dokumentach
\section{Opis projektu}
\label{sec:OpisProjektu}

Celem projektu jest zbudowanie zdalnie sterowanego robota jezdnego. Robot będzie sterowany za pomocą akcelerometru w telefonie. Dane będą przesyłanie za pomocą Wi-Fi lub Bluetooth. Regulacja prędkości będzie się odbywać za pomocą regulatora PID. Dane o prędkości będą pobierane z enkoderów znajdujących się w kołach robota. Opcjonalnie robot będzie wyświetlał szczegółowe dane o swoim stanie wewnętrznym za pomocą wbudowanego w płytkę z mikrokontrolerem wyświetlacza LCD.
\newline
\newline

\begin{figure}[H]
	\centering
	\includegraphics[width=0.8\textwidth]{figures/diagram.png}
	\caption{Architektura systemu}
	\label{fig:Architektura}
\end{figure}

%Obecne we wszystkich dokumentach
\section{Konfiguracja mikrokontrolera}

\begin{figure}[H]
	\centering
	\includegraphics[width=1\textwidth]{figures/schemat-1.jpg}
	\caption{Konfiguracja wyjść mikrokontrolera w programie STM32CubeMX}
	\label{fig:KonfiguracjaMikrokontrolera}
\end{figure}

\newpage
\begin{figure}[H]
	\centering
	\includegraphics[width=0.9\textheight,angle=90]{figures/zegar.png}
	\caption{Konfiguracja zegarów mikrokontrolera}
	\label{fig:KonfiguracjaZegara}
\end{figure}

%Obecne we wszystkich dokumentach
\subsection{Konfiguracja pinów}

\begin{table}[H]
	\centering
	\begin{tabular}{|l|l|l|l|}
		\hline
		PIN & Tryb pracy & Funkcja/etykieta\\
		\hline
		PC14 & OSC32\_IN*	RCC\_OSC32\_IN	&\\
		PC15 & OSC32\_OUT*	RCC\_OSC32\_OUT	&\\
		PH0&  OSC\_IN*	RCC\_OSC\_IN	&\\
		PH1&  OSC\_OUT*&		RCC\_OSC\_OUT	\\
		PA2&	USART2\_TX&	USART\_TX\\
		PA3&	USART2\_RX&	USART\_RX\\
		%PE11&	TIM1\_CH2&	PWM\_SERVO\\
		PA5&	TIM2\_CH1&	PWM\_MOTOR\\
		PB7&    TIM4\_CH2&	PWM\_SERVO\\
		
		
		\hline
	\end{tabular}
	\caption{Konfiguracja pinów mikrokontrolera}
\end{table}

%Obecne we wszystkich dokumentach
\subsection{USART}

Interfejs jest wykorzystywany do komunikacji z modułem Wi-Fi (ESP8266). Moduł odbiera dane za pomocą interfejsu UDP i przekazuje je do mikrokontrolera STM32 za pomocą interfejsu komunikacji szeregowej.

\begin{table}[H]
	\centering
	\begin{tabular}{|l|c|} \hline
		\textbf{Parametr} & Wartość \\
		\hline
		\hline  \textbf{Baud Rate}&115200  \\\hline
		\textbf{Word Length } & \textcolor{blue}{8 Bits (including parity)}\\\hline
		\textbf{Parity} &  None\\
		\hline
		\textbf{Stop Bits}& 1\\
		\hline
	\end{tabular}
	\caption{Konfiguracja peryferium USART}
	\label{tab:USART}
\end{table}

\subsection{Timer 2}

\begin{table}[H]
	\centering
	\begin{tabular}{|l|c|} \hline
		\textbf{Parametr} & Wartość \\
		\hline
		\hline  \textbf{Clock Source}&Internal Clock  \\\hline
		\textbf{Channel1} & PWM Generation CH1\\\hline
		\textbf{Prescaler} & \textcolor{blue}{PWM\_PRESC}\\\hline
		\textbf{Counter Mode} &  Up\\
		\hline
		\textbf{Counter Period}& \textcolor{blue}{PWM\_PERIOD}\\\hline
		\textbf{Internal Clock Division}& No Division\\
		\hline
		\textbf{Mode}& PWM mode 1\\
		\hline
		\textbf{CH Polarity}& High\\
		\hline
	\end{tabular}
	\caption{Konfiguracja peryferium Timer 2}
	\label{tab:Timer2}
\end{table}

\subsection{Timer 2}

\begin{table}[H]
	\centering
	\begin{tabular}{|l|c|} \hline
		\textbf{Parametr} & Wartość \\
		\hline
		\hline  \textbf{Clock Source}&Internal Clock  \\\hline
		\textbf{Channel} & PWM Generation CH2\\\hline
		\textbf{Prescaler} & \textcolor{blue}{999}\\\hline
		\textbf{Counter Mode} &  Up\\
		\hline
		\textbf{Counter Period}& \textcolor{blue}{999}\\\hline
		\textbf{Internal Clock Division}& No Division\\
		\hline
		\textbf{Mode}& PWM mode 1\\
		\hline
		\textbf{CH Polarity}& High\\
		\hline
	\end{tabular}
	\caption{Konfiguracja peryferium Timer 4}
	\label{tab:Timer4}
\end{table}

\subsection{Timer 5}

\begin{table}[H]
	\centering
	\begin{tabular}{|l|c|} \hline
		\textbf{Parametr} & Wartość \\
		\hline
		\hline
		\textbf{Prescaler} & \textcolor{blue}{TIM5\_PRESC}\\\hline
		\textbf{Counter Mode} &  Up\\
		\hline
		\textbf{Counter Period}& \textcolor{blue}{TIM5\_PERIOD}\\\hline
		\textbf{Trigger Event Selection}& \textcolor{blue}{Update Event}\\
		\hline
	\end{tabular}
	\caption{Konfiguracja peryferium Timer 6}
	\label{tab:Timer6}
\end{table}
%Obecne w dokumencie do etapu II oraz III
\section{Urządzenia zewnętrzne}
\textcolor{red}{Rozdział ten powinien zawierać opis i konfigurację %wykorzystanych ukladów
	zewnętrznych, jak np. akcelerometr.}
\subsection{Moduł Wi-Fi ESP8266}
Moduł dobiera dane za pomocą protokołu UDP i przekazuje je odpowiednio sformatowane za pomocą portu szeregowego do mikrokontrolera STM32. Do oprogramowania modułu wykorzystano framework Arduino. \\
Ustawienia komunikacji szeregowej:
	\begin{table}[H]
		\centering
		\begin{tabular}{|l|c|} \hline
			\textbf{Parametr} & Wartość \\
			\hline
			\hline  \textbf{Baud Rate}&115200  \\\hline
			\textbf{Word Length } & 8 Bits (including parity)\\\hline
			\textbf{Parity} &  None\\
			\hline
			\textbf{Stop Bits}& 1\\
			\hline
		\end{tabular}
		\caption{Konfiguracja peryferium UART w module Wi-Fi}
		\label{tab:UARTWiFI}
	\end{table}

%Obecne w dokumencie do etapu II oraz III
\section{Projekt elektroniki}
	\subsection{Połączenie pomiędzy STM32 a modułem WiFi}
		\begin{figure}[H]
			\centering
			\includegraphics[width=0.8\textwidth]{figures/uart.png}
			\caption{Połączenie STM32 z ESP8266}
			\label{fig:UART}
		\end{figure}
	
	\subsection{Schemat połączenia z serwomechanizmem}
	\begin{figure}[H]
		\centering
		\includegraphics[width=1\textwidth]{figures/schemeit-project.png}
		\caption{Schemat elektryczny}
		\label{fig:Schemat elektryczny}
	\end{figure}
	
	\subsection{Schemat połączenia z silnikiem}
	Za pomocą potencjometru regulujemy wypełnienie sygnału PWM. Sygnał ten jest wzmacniany za pomocą tranzystora NPN i przekazywany do silnika DC.
	
	\begin{figure}[H]
		\centering
		\includegraphics[width=0.8\textwidth]{figures/pwm.png}
		\caption{Schemat poglądowy regulacji prędkości obrotowej silnika}
		\label{fig:KonfiguracjaPWM}
	\end{figure}
	
	\subsection{Schemat połączenia z enkoderem}
%Obecne w dokumencie do etapu II oraz III
\section{Konstrukcja mechaniczna}

	\begin{figure}[H]
		\centering
		\includegraphics[width=1\textwidth]{figures/20190410_135905.jpg}
		\caption{Zdjęcie części mechanicznej nr 1}
		\label{fig:Zdjęcie części mechanicznej nr 1}
	\end{figure}
	
		\begin{figure}[H]
		\centering
		\includegraphics[width=1\textwidth]{figures/20190410_135853.jpg}
		\caption{Zdjęcie części mechanicznej nr 2}
		\label{fig:Zdjęcie części mechanicznej nr 2}
	\end{figure}

%Obecne w dokumencie do etapu II oraz III
\section{Opis działania programu}

\subsection{Schemat działania programu}
	\begin{figure}[H]
		\centering
		\includegraphics[width=0.8\textwidth]{figures/diagramPWM.png}
		\caption{Schemat działania programu}
		\label{fig:diagramPWM}
	\end{figure}

\subsection{Funkcja obsługująca przerwanie timera 6}
	\begin{lstlisting}[tabsize=2]
	void HAL_TIM_PeriodElapsedCallback(TIM_HandleTypeDef *htim)
	{
		if(htim->Instance == TIM6)
			HAL_ADC_Start_DMA(&hadc1, (uint32_t *)&adc_value, 1);
	}
	\end{lstlisting}

\subsection{Funkcja obsługująca przerwanie USART}

	\begin{lstlisting}[tabsize=2]
	volatile uint8_t xvalue, yvalue, yprevious, xprevious;
	volatile char xsign;
	void HAL_UART_RxCpltCallback(UART_HandleTypeDef *huart) 
	{
		axis = (char)(Received[nrAxis]);
		sign = (char)(Received[nrSign]);
		value = (uint8_t)(Received[nrValue]);
		
		if(axis == 'x')
		{
			xsign = sign;
			xprevious = xvalue;
			xvalue = value;
		}
		
		if(axis == 'y')
		{
			yprevious = yvalue;
			yvalue = value;
			if(abs(value - xprevious) < 2)
				yvalue = yprevious;
			else if(value > 100)
				yvalue = yprevious;
			else if(value < 50)
				yvalue = yprevious;
		}
		HAL_UART_Receive_DMA(&huart2, &Received, BUFSIZE);
	}
	\end{lstlisting}

%Obecne w dokumencie do etapu II oraz III (jeśli coś zostało niezrealizowane)
\section{Zadania niezrealizowane}
Nie zostało zrealizowane przekazanie napędu z silnika i serwomechanizmu na mechanizm napędowy oraz na ten służący do skręcania. W pierwszym przypadku jest to spowodowane brakiem czasu, wynikający ze zbyt długim poszukiwaniem rozwiązania na problem przeniesienia napędu z silnika do przekładni, natomiast w drugim tym, że wał serwomechanizmu ma stępione zębatki co uniemożliwia przekazanie jakiejkolwiek siły na dalszy podzespół.

%Obecne we wszystkich dokumentach
\section{Podsumowanie}

Udało się zrealizować większość zadań. Nastąpiły drobne zmiany koncepcyjne jak użycie potencjometru do regulacji prędkości obrotowej napędu. To będzie wymagać mniejszej ingerencji gdy będziemy projektować regulator PID.

\newpage
\addcontentsline{toc}{section}{Bibilografia}
\bibliography{bibliografia}
\bibliographystyle{plabbrv}


\end{document}







































