% !TeX encoding = UTF-8
% !TeX spellcheck = pl_PL

% $Id:$

%Author: Wojciech Domski
%Szablon do ząłożeń projektowych, raportu i dokumentacji z steorwników robotów
%Wersja v.1.0.0
%


%% Konfiguracja:
\newcommand{\kurs}{Sterowniki robot\'{o}w}
\newcommand{\formakursu}{Projekt}

%odkomentuj właściwy typ projektu, a pozostałe zostaw zakomentowane
\newcommand{\doctype}{Za\l{}o\.{z}enia projektowe} %etap I
%\newcommand{\doctype}{Raport} %etap II
%\newcommand{\doctype}{Dokumentacja} %etap III

%wpisz nazwę projektu
\newcommand{\projectname}{Humanistycznie upośledzony robot akrobatyczny}

%wpisz akronim projektu
\newcommand{\acronim}{HURA}

%wpisz Imię i nazwisko oraz numer albumu
\newcommand{\osobaA}{Albert \textsc{Lis}, 235534}
%w przypadku projektu jednoosobowego usuń zawartość nowej komendy
\newcommand{\osobaB}{Michał \textsc{Moruń}, 235986}

%wpisz termin w formie, jak poniżej dzień, parzystość, godzina
\newcommand{\termin}{srTP15 ???}

%wpisz imię i nazwisko prowadzącego
\newcommand{\prowadzacy}{mgr in\.{z}. Wojciech \textsc{Domski}}

\documentclass[10pt, a4paper]{article}

\usepackage[usenames,dvipsnames]{xcolor}

\include{preambula}
	
\begin{document}

\def\tablename{Tabela}	%zmienienie nazwy tabel z Tablica na Tabela

\begin{titlepage}
	\begin{center}
		\textsc{\LARGE \formakursu}\\[1cm]		
		\textsc{\Large \kurs}\\[0.5cm]		
		\rule{\textwidth}{0.08cm}\\[0.4cm]
		{\huge \bfseries \doctype}\\[1cm]
		{\huge \bfseries \projectname}\\[0.5cm]
		{\huge \bfseries \acronim}\\[0.4cm]
		\rule{\textwidth}{0.08cm}\\[1cm]
		
		\begin{flushright} \large
		\emph{Skład grupy:}\\
		\osobaA\\
		\osobaB\\[0.4cm]
		
		\emph{Termin: }\termin\\[0.4cm]

		\emph{Prowadzący:} \\
		\prowadzacy \\
		
		\end{flushright}
		
		\vfill
		
		{\large \today}
	\end{center}	
\end{titlepage}

\newpage
\tableofcontents
\newpage

%Obecne we wszystkich dokumentach
\textcolor{Red}{To musi się znaleźć:} \newline
\textcolor{Red}{k1 in [0,1.0] — poprawne opracowanie dokumentu w systemie składania tekstu LaTeX, wykorzystanie dostarczonego szablonu \newline
	k2 in [0,0.5] — przynajmniej dwie pozycje literaturowe traktujące o problematyce projektu \newline
	k3 in [0,0.5] — przynajmniej 2 pozycje ściśle związane z wykorzystanym sprzętem, układami elektronicznymi,  modułami, itp.\newline
	k4 in [0,1.5] — merytoryczna część założeń projektowych \newline
	k5 in [0,0.5] — podział prac w projekcie na zadania.\newline}
\section{Opis projektu}
\label{sec:OpisProjektu}

Celem projektu jest zbudowanie zdalnie sterowanego robota jezdnego. Robot będzie sterowany za pomocą akcelerometru w telefonie. Dane będą przesyłanie za pomocą Wi-Fi lub Bluetooth. Regulacja prędkości będzie się odbywać za pomocą regulatora PID. Dane o prędkości będą pobierane z enkoderów znajdujących się w kołach robota. Opcjonalnie robot będzie wyświetlał szczegółowe dane o swoim stanie wewnętrznym za pomocą wbudowanego w płytkę z mikrokontrolerem wyświetlacza LCD.
\newline
\newline

\begin{figure}[H]
	\centering
	\includegraphics[width=0.8\textwidth]{figures/diagram.png}
	\caption{Architektura systemu}
	\label{fig:Architektura}
\end{figure}

\section{Założenia projektowe}

	\subsection{Mechanika}
	\begin{enumerate}
		\item Napęd
		\newline
		Napęd będzie realizowany na tylną oś za pomocą silnika szczotkowego DC. Regulacja prędkości oparta o regulator PID oraz sterowanie PWM.
		
		\item Sterowanie
		\newline
		Skręcanie będzie oparte o serwomechanizm. Serwomechanizm realizuje skręt przednich kół za pomocą poprzecznej belki przymocowanej do kół.
		
		\item Rama
		\newline
		Rama zbudowana z klocków lego. Posiada duże możliwości dopasowania do zmian w trakcie projektu.
	\end{enumerate}

	\subsection{Elektronika}
	\begin{enumerate}
		\item Mikrokontroler
		\newline
		Sterownik dostarczony przez prowadzącego STM32L476GDiscovery.
		
		\item Pomiar prędkości
		\newline
		Realizowany za pomocą enkoderów znajdujących się w kołach robota.
		
		\item Zasilanie
		\newline
		Oparte o akumulatory li-ion 18650 lub powerbank. Dopasowanie napięcia za pomocą przetwornicy step-up MT3608 do napędu kół oraz step-down do zasilania mikrokontrolera i modułu Wi-Fi w standardzie 3.3V.
	\end{enumerate}

	\subsection{Komunikacja}
	\begin{enumerate}
		\item Połączenie ze smartfonem
		\newline
		Realizowane za pomocą modułu Wi-Fi ESP8266. W telefonie do komunikacji posłuży aplikacja RoboRemo.
		\item Połączenie modułu Wi-Fi z mikrokontrolerem
		\newline
		Realizowane za pomocą portu szeregowego.
	\end{enumerate}
	

%Obecne w dokumencie do etapu I
\section{Harmonogram pracy}

\subsection{Zakres prac}
	\begin{enumerate}
		\item Zapoznanie się z mikrokontrolerem
		\newline
		Wykorzystane to tego celu zostaną poradniki ze strony www.forbot.pl. \cite{kurs1, kurs2, kurs3}
	\end{enumerate}
\subsection{Kamienie milowe}
	\begin{enumerate}
		\item Implementacja działającego prototypu sterowanego joystickiem na płytce.
		\item Implementacja regulacji prędkości w oparciu o regulator PID.
		\item Implementacja sterowania smartfonem.
	\end{enumerate}

\subsection{Wykres Gantta}
Należy wstawić diagram Gantta oraz określić ścieżkę 
krytyczną. Ponadto zaznaczyć i opisać kamienie milowe.

\begin{figure}[H]
	\centering
	\includegraphics[width=0.5\textwidth]{figures/obraz.png}
	\caption{Diagram Gantta}
	\label{fig:DiagramGantta}
\end{figure}

%Obecne w dokumencie do etapu I
\subsection{Podział pracy}

\textbf{Każdy z członków grupy powinien w każdym etapie mieć wymienione od 2 do 4 zadań.}
Przykładowa tabele podziału zadań dla etapu II 
(Tab. \ref{tab:PodzialPracyEtap2}) oraz dla etapu III 
(Tab. \ref{tab:PodzialPracyEtap3})
zostały przedstawione poniżej. 
Przy podziale prac nie uwzględniamy tworzenia dokumentacji projektu!

Przykładowy podział prac:

\begin{table}[H]
	\centering
	\begin{tabular}{|L{7cm}|L{0.8cm}||L{7cm}|L{0.8cm}|}
		\hline
		\hline
		\textbf{Albert Lis} & 
		\% & 
		\textbf{Michał Moruń} & \%\\
		\hline
		\hline
		Schemat elektryczny i elektroniczny		& &	
		Schemat mechaniczny &\\
		\hline
		Budowanie odpowiednich algorytmów & &
		Budowanie odpowiednich algorytmów &\\
		\hline
		Budowa modułu elektronicznego & &
		Budowa modułu mechanicznego & \\
		\hline
		Integracja części mechanicznej oraz elektronicznej & & 
		Integracja części mechanicznej oraz elektronicznej &\\
		\hline
	\end{tabular}
	\caption{Podział pracy -- Etap II}
	\label{tab:PodzialPracyEtap2}
\end{table}

\begin{table}[H]
	\centering
	\begin{tabular}{|L{7cm}|L{0.8cm}||L{7cm}|L{0.8cm}|}
		\hline
		\hline
		\textbf{Albert Lis} & 
		\% & 
		\textbf{Michał Moruń} & \%\\
		\hline
		\hline
		Utworzenie modułu integrującego robota z telefonem & &	
		Utworzenie modułu integrującego robota z telefonem &\\
		\hline
		Integracja ze sobą wszystkich modułów & &
		Integracja ze sobą wszystkich modułów & \\
		\hline
		 Stworzenie interfejsu użytkownika & &
		 Stworzenie interfejsu użytkownika &\\
		\hline
	\end{tabular}
	\caption{Podział pracy -- Etap III}
	\label{tab:PodzialPracyEtap3}
\end{table}



\newpage
%\addcontentsline{toc}{section}{Bibilografia}
%\bibliography{bibliografia}
%\bibliographystyle{plabbrv}
\begin{thebibliography}{9}
	
	\bibitem{kurs1}
	\href{https://forbot.pl/blog/kurs-stm32-f4-1-czas-poznac-hal-spis-tresci-kursu-id14114}{Kurs STM32 F4 z wykorzystaniem HAL oraz Cube}
	\bibitem{kurs2}
	\href{https://forbot.pl/blog/stm32-praktyce-1-platforma-srodowisko-id2733}{Kurs STM32 F1 z wykorzystaniem bibliotek STDPeriph}
	\bibitem{kurs3}
	\href{https://forbot.pl/blog/kurs-stm32-f1-migracja-na-hal-wstep-spis-tresci-id23580} {Kurs STM32 F1 z wykorzystaniem bibliotek HAL}
		\bibitem{kurs4}
	\href{https://media.readthedocs.org/pdf/arduino-esp8266/docs_to_readthedocs/arduino-esp8266.pdf} {ESP8266 Arduino Core Documentation}
			\bibitem{kurs5}
	\href{https://docer.pl/doc/n5xcex8} {Teoria sterowania w ćwiczeniach}
	
\end{thebibliography}


\end{document}







































